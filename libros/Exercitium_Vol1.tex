% Exercitium_Vol1.tex
% Exercitium (curso 2013-14)
% José A. Alonso Jiménez <https://jaalonso.github.io>
% Sevilla, 14 de Octubre de 2025
% ======================================================================

\documentclass[a4paper,12pt,twoside]{book}

%%%%%%%%%%%%%%%%%%%%%%%%%%%%%%%%%%%%%%%%%%%%%%%%%%%%%%%%%%%%%%%%%%%%%%%%%%%%%%
%% § Paquetes adicionales                                                   %%
%%%%%%%%%%%%%%%%%%%%%%%%%%%%%%%%%%%%%%%%%%%%%%%%%%%%%%%%%%%%%%%%%%%%%%%%%%%%%%

% Configuración para XeLaTeX
\usepackage{tocloft}
\setlength{\cftbeforechapskip}{2ex}
\setlength{\cftbeforesecskip}{0.5ex}
\setlength{\cftchapnumwidth}{2em}
\setlength{\cftsecnumwidth}{14mm}
\cftsetindents{chapter}{0em}{\cftchapnumwidth}

\usepackage{fontspec}
\usepackage{xunicode}
\usepackage{xltxtra}
\defaultfontfeatures{Ligatures=TeX,Numbers=OldStyle} % inicial
\setromanfont{DejaVu Sans} % inicial
\setmonofont{DejaVu Sans Mono}[Scale={0.90}] % inicial

% \setsansfont{Arial}

% \setmainfont{DejaVu Math TeX Gyre}
% \setmonofont{DejaVu Math TeX Gyre}[Scale={1.1}]

% Notas: La lista de fuentes disponibles se obtiene con fc-list

\usepackage[spanish]{babel}          % Castellanización.
\usepackage{a4wide}
\usepackage{amssymb, amsmath, mathtools}

\usepackage{newunicodechar}
\newunicodechar{⟹}{\ensuremath{\mathnormal\Longrightarrow}}
\usepackage[dvipsnames]{xcolor}

\usepackage{minted}
\usemintedstyle{emacs}
\definecolor{bg}{rgb}{0.95,0.95,0.95}
\newminted{lean}{bgcolor=bg, autogobble=true}

\setminted{fontsize=\small,baselinestretch=1}

\newcommand{\ejercicio}[1]{\inputminted[firstline=7]{haskell}{../src/#1.hs}}

\linespread{1.05}                  % Distancia entre líneas
\setlength{\parindent}{2em}        % Indentación de comienzo de párrafo
% \deactivatetilden                  % Elima uso de ~ para la eñe
\raggedbottom                      % No ajusta los espacios verticales.

\usepackage[%
  colorlinks=true,
  urlcolor=blue,
  % pdftex,
  pdfauthor={José A. Alonso <jalonso@us.es>},%
  pdftitle={Exercitium Vol. 1},%
  pdfstartview=FitH,%
  bookmarks=false]{hyperref}

\setcounter{tocdepth}{1}
\setcounter{secnumdepth}{4}

% No indenta los item
\usepackage{enumitem}
\setlist[itemize]{leftmargin=*}

%%%%%%%%%%%%%%%%%%%%%%%%%%%%%%%%%%%%%%%%%%%%%%%%%%%%%%%%%%%%%%%%%%%%%%%%%%%%%%
%% § Cabeceras                                                              %%
%%%%%%%%%%%%%%%%%%%%%%%%%%%%%%%%%%%%%%%%%%%%%%%%%%%%%%%%%%%%%%%%%%%%%%%%%%%%%%

\usepackage{fancyhdr}

\addtolength{\headheight}{\baselineskip}

\pagestyle{fancy}

\cfoot{}

%\fancyhead{}
%\fancyhead[RE]{\mdseries\sffamily \nouppercase{\leftmark}}
%\fancyhead[LO]{\mdseries\sffamily \nouppercase{\rightmark}}
%\fancyhead[LE,RO]{\mdseries\sffamily \thepage}

% \fancyhead{}
% \fancyhead[RE]{\mdseries \nouppercase{\leftmark}}
% \fancyhead[LO]{\mdseries \nouppercase{\rightmark}}
% \fancyhead[LE,RO]{\mdseries \thepage}

\fancyhead{}
\fancyhead[RE,LO]{\mdseries \nouppercase{\leftmark}}  % Capítulo en ambos lados
\fancyhead[LE,RO]{\mdseries \thepage}                 % Página en esquinas

%%%%%%%%%%%%%%%%%%%%%%%%%%%%%%%%%%%%%%%%%%%%%%%%%%%%%%%%%%%%%%%%%%%%%%%%%%%%%%
%% § Título                                                                 %%
%%%%%%%%%%%%%%%%%%%%%%%%%%%%%%%%%%%%%%%%%%%%%%%%%%%%%%%%%%%%%%%%%%%%%%%%%%%%%%

\title{
  {\LARGE Exercitium \\
    {\Large Ejercicios de programación funcional con Haskell \\
      {\normalsize (Volumen 1: curso 2013--14)
    }}}}
\author{
  \href{https://jaalonso.github.io/}{José A. Alonso Jiménez}}
\date{\vfill \hrule \vspace*{2mm}
  \begin{tabular}{l}
      Grupo de Lógica Computacional \\
      Dpto. de Ciencias de la Computación e Inteligencia Artificial \\
      Universidad de Sevilla  \\
    Sevilla, 14 de octubre de 2025
  \end{tabular}\hfill\mbox{}}

%%%%%%%%%%%%%%%%%%%%%%%%%%%%%%%%%%%%%%%%%%%%%%%%%%%%%%%%%%%%%%%%%%%%%%%%%%%%%%
%% § Documento                                                              %%
%%%%%%%%%%%%%%%%%%%%%%%%%%%%%%%%%%%%%%%%%%%%%%%%%%%%%%%%%%%%%%%%%%%%%%%%%%%%%%

% \includeonly{}

% \includexmp{licencia}

\begin{document}

\maketitle
\newpage

\input{Licencia/licenciaCC}
\newpage

\tableofcontents
\clearpage

\renewcommand{\chaptername}{Ejercicio}

\chapter*{Introducción}

% \mbox{} \hspace*{1cm}

\begin{quote}
  ``\textit{The chief goal of my work as an educator and author is to
  help people learn to write beautiful programs.}''

  (Donald Knuth en
  \href{http://www.paulgraham.com/knuth.html}{Computer programming as an art})
\end{quote}

\vspace* {1cm}

Este libro es una recopilación de las soluciones de los ejercicios
propuestos en el blog
\href{https://jaalonso.github.io/exercitium}
     {Exercitium}\
     \footnote{\url{https://jaalonso.github.io/exercitium}}
durante el curso 2013--14.

El principal objetivo de Exercitium es servir de complemento a la
asignatura de
\href{https://jaalonso.github.io/cursos/i1m-13}
     {Informática}\
     \footnote{\url{https://jaalonso.github.io/cursos/i1m-13}}
de 1º del Grado en Matemáticas de la Universidad de Sevilla.

Con los problemas de Exercitium, a diferencias de los de las
\href{https://web.archive.org/web/20250614171615/https://www.cs.us.es/~jalonso/cursos/i1m-13/ejercicios/ejercicios-I1M-2013.pdf}
     {relaciones}\
     \footnote{\url{https://web.archive.org/web/20250614171615/https://www.cs.us.es/~jalonso/cursos/i1m-13/ejercicios/ejercicios-I1M-2013.pdf}},
se pretende practicar con los conocimientos adquiridos durante todo el
curso, mientras que con las relaciones están orientadas a los nuevos
conocimientos.

En cada ejercicio se muestra distintas soluciones, se comprueba con
QuickCheck su equivalencia y se compara su eficiencia.

El código de los ejercicios del libro se encuentra en
\href{https://github.com/jaalonso/Exercitium}
     {GitHub}\
     \footnote{\url{https://github.com/jaalonso/Exercitium}}

% \chapter{Ejercicios de Abril de 2014}

\chapter{Iguales al siguiente}
\ejercicio{Iguales_al_siguiente}

\chapter{Ordenación por el máximo}
\ejercicio{Ordenados_por_maximo}

\chapter{La bandera tricolor}
\ejercicio{Bandera_tricolor}

\chapter{Determinación de los elementos minimales}
\ejercicio{ElementosMinimales}

\chapter{Mastermind}
\ejercicio{Mastermind}

\chapter{Primos consecutivos con media capicúa}
\ejercicio{Primos_consecutivos_con_media_capicua}

\chapter{Anagramas}
\ejercicio{Anagramas}

\chapter{Primos equidistantes}
\ejercicio{Primos_equidistantes}

% \chapter{Ejercicios de Mayo de 2014}

\chapter{Suma si todos los valores son justos}
\ejercicio{Suma_si_todos_justos}

\chapter{Matriz de Toeplitz}
\ejercicio{Matriz_Toeplitz}

\chapter{Máximos locales}
\ejercicio{Maximos_locales}

\chapter{Lista cuadrada}
\ejercicio{Lista_cuadrada}

\chapter{Segmentos maximales con elementos consecutivos}
\ejercicio{Segmentos_consecutivos}

\chapter{Valores de polinomios representados con vectores}
\ejercicio{Valor_de_un_polinomio}

\chapter{Ramas de un árbol}
\ejercicio{Ramas_de_un_arbol}

\chapter{Alfabeto comenzando en un carácter}
\ejercicio{Alfabeto_desde}

\chapter{Numeración de ternas}
\ejercicio{Numeracion_de_ternas}

\chapter{Ordenación de estructuras}
\ejercicio{Ordenacion_de_estructuras}

\chapter{Emparejamiento binario}
\ejercicio{Emparejamiento_binario}

\chapter{Amplia columnas}
\ejercicio{Amplia_columnas}

\chapter{Regiones determinadas por n rectas del plano}
\ejercicio{Regiones}

\chapter{Elemento más repetido de manera consecutiva}
\ejercicio{Mas_repetido}

\chapter{Número de pares de elementos adyacentes iguales en una matriz}
\ejercicio{Pares_adyacentes_iguales}

\chapter{Mayor producto de las ramas de un árbol}
\ejercicio{Mayor_producto_de_las_ramas_de_un_arbol}

\chapter{Biparticiones de una lista}
\ejercicio{Biparticiones_de_una_lista}

\chapter{Trenzado de listas}
\ejercicio{Trenzado_de_listas}

\chapter{Números triangulares con n cifras distintas}
\ejercicio{Triangulares_con_cifras}

\chapter{Enumeración de árboles binarios}
\ejercicio{Enumera_arbol}

\chapter{Algún vecino menor}
\ejercicio{Algun_vecino_menor}

\chapter{Reiteración de una función}
\ejercicio{Reiteracion_de_funciones}

% \chapter{Ejercicios de Junio de 2014}

\chapter{Pim, Pam, Pum y divisibilidad}
\ejercicio{PimPamPum}

\chapter{Código de las alergias}
\ejercicio{Alergias}

\chapter{Índices de valores verdaderos}
\ejercicio{Indices_verdaderos}

\chapter{Descomposiciones triangulares}
\ejercicio{Descomposiciones_triangulares}

\chapter{Número de inversiones}
\ejercicio{Numero_de_inversiones}

\chapter{Separación por posición}
\ejercicio{Separacion_por_posicion}

\chapter{Emparejamiento de árboles}
\ejercicio{Emparejamiento_de_arboles}

\chapter{Eliminación de las ocurrencias aisladas}
\ejercicio{Elimina_aisladas}

\chapter{Ordenada cíclicamente}
\ejercicio{Ordenada_ciclicamente}

\chapter{Órbita prima}
\ejercicio{Orbita_prima}

\chapter{Divisores de un número con final dado}
\ejercicio{Divisores_con_final}

\chapter{Descomposiciones de x como sumas de n sumandos de una lista}
\ejercicio{Descomposiciones_con_n_sumandos}

\chapter{Selección hasta el primero que falla inclusive}
\ejercicio{Seleccion_con_fallo}

\chapter{Buscaminas}
\ejercicio{Buscaminas}

\chapter{Mayor sucesión del problema 3n+1}
\ejercicio{Mayor_sucesion_3n_mas_1}

\chapter{Filtro booleano}
\ejercicio{Filtro_booleano}

\chapter{Entero positivo de la cadena}
\ejercicio{Entero_positivo_de_la_cadena}

\chapter{N gramas}
\ejercicio{N_gramas}

\chapter{Sopa de letras}
\ejercicio{Sopa_de_letras}

\chapter{Intercalación de n copias}
\ejercicio{Intercala_n_copias}

\chapter{Eliminación de n elementos}
\ejercicio{Elimina_n_elementos}

% \chapter{Ejercicios de Julio de 2014}

\chapter{Límite de sucesiones}
\ejercicio{Limites_de_sucesiones}

\chapter{Empiezan con mayúscula}
\ejercicio{Empiezan_con_mayuscula}

\chapter{Renombramiento de un árbol}
\ejercicio{Renombra_arbol}

\chapter{Divide si todos son múltiplos}
\ejercicio{Divide_si_todos_multiplos}

\chapter{Ventana deslizante}
\ejercicio{Ventana_deslizante}

\chapter{Representación de Zeckendorf}
\ejercicio{Representacion_de_Zeckendorf}

\chapter{Elemento más cercano que cumple una propiedad}
\ejercicio{Codigo_Morse}

\chapter{Producto cartesiano de una familia de conjuntos}
\ejercicio{Producto_cartesiano}

\chapter{Todas tienen par}
\ejercicio{Todas_tienen_par}

\chapter{Sucesiones pucelanas}
\ejercicio{Sucesiones_pucelanas}

\chapter{Producto de matrices como listas de listas}
\ejercicio{Producto_de_matrices_como_listas_de_listas}

\chapter{Inserción en árboles binarios de búsqueda}
\ejercicio{Insercion_en_arboles_binarios_de_busqueda}

\chapter{Matriz permutación}
\ejercicio{Matriz_permutacion}

\chapter{Números con todos sus dígitos primos}
\ejercicio{Numeros_con_digitos_primos}

\chapter{Cadenas de ceros y unos}
\ejercicio{Cadenas0y1}

\chapter{Clausura de un conjunto respecto de una función}
\ejercicio{Clausura}

\chapter{Sustitución en una expresión aritmética}
\ejercicio{Sustitucion_en_una_expresion_aritmetica}

\chapter{Laberinto numérico}
\ejercicio{Laberinto_numerico}

\end{document}

%%% Local Variables:
%%% mode: latex
%%% TeX-master: t
%%% End:
